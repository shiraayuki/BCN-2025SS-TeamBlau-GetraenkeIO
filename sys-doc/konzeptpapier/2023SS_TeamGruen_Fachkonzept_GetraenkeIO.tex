\PassOptionsToPackage{type=CC,modifier=by,version=4.0}{doclicense}
\documentclass[conference,a4paper]{cs-techrep}
\pdfoutput=1 % pdflatex hint for arxiv.org (within first 5 lines)

% Class cs-techrep.cls loads biblatex / biber with predefined options
\addbibresource{embedded.bib}       % its content is declared below, embedded within this tex-file
\addbibresource{webdev_commons.bib} % includes REST, React, Angular, Vue, Svelte, Docker, AWS-*, Socket.IO, and many more!
\addbibresource{cpn_all_all.bib}    % includes all previous CyberLytics@OTH-AW technical reports

% ======================================================================
% EDIT THESE:

\cstechrepAuthorListTex{Eduard Taach, Justin Weinhut, Martin Dotzler, Nikolas Wegerer, Andreas Ehrles, Christoph P.\ Neumann\,\orcidlink{0000-0002-5936-631X}}
\cstechrepAuthorListBib{Eduard Taach und Justin Weinhut und Martin Dotzler und Nikolas Wegerer und Andreas Ehrles und Christoph P. Neumann}

% Capitalization: https://capitalizemytitle.com/style/Chicago/
\cstechrepTitleTex{GetraenkeIO: Eine Cloud-Native Getränkeverwaltungssoftware mit Bestandsaufnahme}
 % IF you need manual linebreaks in the titel, then clone the title without linebreaks for BibTeX:
\cstechrepTitleBib{{\cstechrepTitleTex}}

%\cstechrepDepartment{CyberLytics\-/Lab at the Department of Electrical Engineering, Media, and Computer Science}%
\cstechrepDepartment{CyberLytics\-/Lab an der Fakultät Elektrotechnik, Medien und Informatik} % DE
\cstechrepInstitution{Ostbayerische Technische Hochschule Amberg\-/Weiden}
%\cstechrepAddress{Amberg, Germany}
\cstechrepAddress{Amberg, Deutschland} % DE
%\cstechrepType{Concept Paper}
\cstechrepType{Konzeptpapier} % DE
\cstechrepYear{2025}
\cstechrepMonth{5}
\cstechrepNumber{CL-CP-\cstechrepYear{}-42}
%\cstechrepLang{english}  % en-US
\cstechrepLang{ngerman} % DE

% Special remark on babel/csquotes terminology in regard with US-vs-UK:
% en-US  = [english]/[american]/[usenglish] (+ [canadian])
% en-UK  =           [british] /[ukenglish] (+ [australian]) <OXFORD>
% For cs-techrep (like ACM), the recommended english variant is en-US!

% DO NOT DELETE THIS:
\filecontentsForceExpansion|[] % force command expansion inside a filecontents* environment
\begin{filecontents*}[overwrite]{selfref.bib}
    @TECHREPORT{selfref,
        author = {|cstechrepAuthorListBib},
        title  = {\cstechrepTitleBib},
        institution = {\cstechrepInstitution, \cstechrepDepartment},
        type   = {\cstechrepType},
        number = {\cstechrepNumber},
        year   = {|cstechrepYear},
        month  = {|cstechrepMonth},
        langid  = {|cstechrepLang},
    }
\end{filecontents*}

% ======================================================================
% EDIT THIS:

\begin{filecontents}[overwrite]{embedded.bib}
@online{ieee2015howto,
    author = {Michael Shell},
    title = {How to Use the {IEEEtran} \LaTeX\ Class},
    url = {http://mirrors.ctan.org/macros/latex/contrib/IEEEtran/IEEEtran_HOWTO.pdf},
    year = {2015}
}
@online{ieee2018formattingrules,
    author = {{IEEE}},
    title = {Conference Template and Formatting Specifications},
    url = {https://www.ieee.org/content/dam/ieee-org/ieee/web/org/conferences/Conference-template-A4.doc},
    year = {2018}
}
@online{iaria2014formattingrules,
    author = {{IARIA}},
    title = {Formatting Rules},
    url = {http://www.iaria.org/formatting.doc},
    year = {2014}
}
@online{iaria2009editorialrules,
    _author = {Cosmin Dini},
    author = {{IARIA}},
    title = {Editorial Rules},
    url = {https://www.iaria.org/editorialrules.html},
    year = {2009}
}
@online{languagetool,
    author = {{LanguageTooler GmbH}},
    title  = {{LangueTool}},
    url    = {https://languagetool.org/overleaf}
}
@online{overleaf,
    author = {{Digital Science UK Limited}},
    title  = {{Overleaf}},
    url    = {https://www.overleaf.com}
}
\end{filecontents}

\usepackage{fontawesome} % i.a., \faWarning{}
\usepackage{relsize}     % i.a., \textsmaller{...}
\usepackage{lipsum}      % for blindtext

% For the types of users diagram (like an organigram / org chart)
\usepackage[edges]{forest} % = pgf/TikZ-based package for drawing trees
\usetikzlibrary{fit}

% ======================================================================

% Prevent a page break before an itemize list, because we will use them
% for user stories / acceptance criteria:
\makeatletter
\@beginparpenalty=10000
\makeatother

% cf. https://ctan.org/pkg/acronym
% Usage:
% singular, within sentence       = \ac{gui}
% singular, beginning of sentence = \Ac{gui}
% plural, within sentence         = \acp{gui}
% plural, beginning of sentence   = \Acp{gui}
\begin{acronym}
    \acro{gui}[GUI]{Graphical User Interface}
    \acro{ide}[IDE]{Integrated Development Environment}
\end{acronym}

% https://www.silbentrennung24.de/
% https://www.hyphenation24.com/
\hyphenation{block-chain block-chains Ethe-re-um}

\begin{document}
\selectlanguage{\cstechrepLang}

\maketitle

\begin{abstract}
Bla Bla Bla TODO!!!
\{\,\faWarning{}The abstract does neither mention a teaching module nor a team/project,
it is a summary of the content, thus, the functional objective -- and maybe the intended technology stack.
Do NOT remove the abstract \faWarning{}, this section is mandatory.
You should consider comparing your self-written abstract with the result of a generative AI that summarizes your content after you have written a nearly stable draft version. However, do not use a verbatim copy to replace your abstract, just use generative AI for inspirational purposes.\}
\end{abstract}

% A list of IEEE Computer Society appoved keywords can be obtained at
% http://www.computer.org/mc/keywords/keywords.htm
\begin{IEEEkeywords}
template; lorem ipsum.
\end{IEEEkeywords}

\section{Einleitung \textbar{} Background and Motivation \textbar{} Mission Statement \textbar{} Elevator Pitch}

The cs-techrep formatting is adopted both from \textsmaller{IEEE} \cite{ieee2018formattingrules} and \textsmaller{IARIA} \cite{iaria2014formattingrules} styles.
The cs-techrep \LaTeX\ class is based on \textsmaller{IEEE}tran class \cite{ieee2015howto}.
In addition, be aware of the supplementary \textsmaller{IARIA} editorial rules \cite{iaria2009editorialrules} \faWarning{} that provide a beginner-friendly set of further advices.
It is recommended to use a grammar tool, e.\,g., the \texttt{LanguageTool} \cite{languagetool} browser plugin in combination with \texttt{Overleaf} \cite{overleaf}.

The title of your paper should not exceed two lines \faWarning{}. In exceptional cases, three lines might be allowed. A four-line-title is absolutely forbidden (hint: use the longer form in the abstract).

For capitalization of titles and section headings, use a web tool like \href{https://capitalizemytitle.com/style/Chicago/}{\texttt{Capitalize My Title}} \faWarning{} with the option \texttt{Chicago} for capitalization rules by Chicago Manual of Style (\textsmaller{CMOS}).

The pipe symbol \textquote{\textbar{}} in the section headings represents alternatives! Choose one and remove the others \faWarning{}. The selectively provided quoted terms are special German alternatives. You may deviate from the structure of this example document and its exemplary section headings.

The introduction needs to be written from perspective of a subject-matter expert \faWarning{} and NOT from a technical perspective. Provide the USPs of your intended software product (in German: \textquote{fachliche Alleinstellungsmerkmale}).

\section{Optional: Related Work}
TODO: löschen???

\section{Funktionelles Design}
\subsection{Verfügbare Rollen}
Die User Stories basieren auf zwei verschiedenen Rollen:
\begin{enumerate}
\item \textbf{Trinker}: 
Der Reguläre Benutzer des Systems, der in einem Vereinsheim o.ä. ab und zu ein Bierchen trink und dieses im System buchen möchte, damit er seine Zeche bezahlen kann.
\item \textbf{Getränkeverwalter}: 
Der Getränkeverwalter/Getränkewart des Vereinsheims, der sich um den Getränkenachschub kümmert und die regelmäßig das Geld der Trinker eintreibt.
\end{enumerate}

\section{Functional Requirements \textbar{}\\User Stories \textbar{} Use Cases}

Die Anforderungen an das Projekt werden im Folgenden als User-Storys formuliert, wobei die MUSS-Anforderung am Ende des Sprints das MVP bilden. Falls noch Zeit bleibt sollen im weiteren die SOLL-Anforderungen und kann-Anforderungen umgesetzt werden.

\subsection{MUSS-Anforderungen}

\begin{enumerate}[{USM}1]
	
\item \textbf{Registrieren}: Ich als Benutzer möchte mich registrieren, damit ich im System bekannt bin.
	\begin{enumerate}
	\item Eingabe von Benutzername und Passwort durch den Benutzer.
	\item Benutzer Interagiert mit Web-Oberfläche
	\item Ausgabe Meldung an den Benutzer das es geklappt hat.
	\item Wenn nicht erfolgreich => Ausgabe einer Fehlermeldung
	\end{enumerate}

\item \textbf{Einloggen}: Ich als Benutzer möchte mich einloggen können, weil ich die Seite nutzen möchte um Getränke zu buchen.
\begin{enumerate}
	\item Vorbedingung: Benutzer ist registriert.
	\item Eingabe von Benutzername und Passwort durch den Benutzer.
	\item Benutzer Interagiert mit Web-Oberfläche
	\item Wenn der Benutzername und das Passwort mit einem registriertem Benutzer übereinstimmen => weiterleiten auf Übersichtsseite
	\item Wenn nicht erfolgreich => Ausgabe einer Fehlermeldung
\end{enumerate}

\item \textbf{Übersicht}: Ich als Trinker möchte eine Übersicht über alle verfügbare Getränke, weil ich sehen möchte, welches Getränk ich kaufen kann.
\begin{enumerate}
	\item Vorbedingung: Benutzer ist eingeloggt.
	\item Benutzer sieht alle verfügbaren Getränke.
	\item Benutzer sieht den Preis der verfügbaren Getränke.
	\item Benutzer Interagiert mit Web-Oberfläche
\end{enumerate}

\item \textbf{Getränk buchen}: Ich als Trinker möchte ein verfügbares Getränk buchen, damit dieses als von mir Getrunken im System hinterlegt wird.
\begin{enumerate}
	\item Vorbedingung: Benutzer ist eingeloggt.
	\item Benutzer kann auf ein Getränk klicken.
	\item Benutzer Getränke verfügbar => Das Guthaben des Benutzers verringert sich um den Preis des Getränks.
	\item Getränk nicht verfügbar => Benutzer kann das Getränk nicht drücken bzw. wird durch eine Meldung darauf hingewiesen.
\end{enumerate}

\item \textbf{Guthaben aufladen}: Ich als Getränkeverwalter möchte das Guthaben der registrierten Mitglieder verändern/aufladen können, damit sie bei mir für ihre Getränke bezahlen können.
\begin{enumerate}
	\item Vorbedingung: Benutzer mit Berechtigung Getränkeverwalter ist eingeloggt.
	\item Getränkeverwalter kann ein Menü zur Guthabenverwaltung öffnen
	\item Getränkeverwalter kann das Guthaben von Trinkern anpassen
	\item Nach der Anpassung des Guthabens ist das veränderte Guthaben beim Betroffenen Trinker sichtbar und verwendbar.
\end{enumerate}

\item \textbf{Getränkedaten Verwalten}: Ich als Getränkeverwalter möchte einstellen können, welche Getränke zu welchen Preisen verfügbar sind, damit diese von Trinkern gekauft werden können.
\begin{enumerate}
	\item Vorbedingung: Benutzer mit Berechtigung Getränkeverwalter ist eingeloggt.
	\item Getränkeverwalter kann ein Menü zur Getränkeverwaltung öffnen
	\item Getränkeverwalter kann Getränke und den Zugehörigen Preis anpassen.
	\item Nach der Anpassung der Getränke sind diese bei Trinkern sichtbar und buchbar.
\end{enumerate}

\end{enumerate}

\subsection{SOLL-Anforderungen}

\begin{enumerate}[{USS}1]

\item \textbf{Getränkeverlauf anzeigen}: Ich als Trinker möchte sehen können, welche Getränke ich wann gekauft habe, weil ich das auch später noch nachvollziehen möchte.
\begin{enumerate}
	\item Vorbedingung: Benutzer mit Berechtigung Trinker ist eingeloggt.
	\item Benutzer öffnet ein Menü in dem die Historie angezeigt wird.
	\item Alle von einem Benutzer getrunkenen Getränke werden in einer Liste angezeigt.
	\item Der Benutzer kann den jeweiligen Preis zum Kaufdatum der Getränke in der Historie sehen
\end{enumerate}

\item \textbf{Bestandsverwaltung}: Ich als Getränkeverwalter möchte den verfügbaren Bestand von Getränken in der Anwendung hinterlegen können, weil ich sehe möchte, wann sie leer werden.
\begin{enumerate}
	\item Vorbedingung: Benutzer mit Berechtigung Getränkeverwalter ist eingeloggt.
	\item Benutzer öffnet ein Menü in dem er die anzahl der verfügbaren Getränke einstellen kann.
	\item Nach speichern/hinzufügen der anzahl der verfügbaren Getränke wird diese in die Datenbank übernommen.
	\item Wenn Trinker ein Getränk trinken, wird der Bestand des jeweiligen Getränks um 1 reduziert.
\end{enumerate}

\item \textbf{Umsatzübersicht}: Ich als Getränkeverwalter möchte den Umsatz den jedes Mitglied generiert hat einsehen können, damit ich auswerten kann wer in einem bestimmten Zeitraum am meisten Getrunken hat.
\begin{enumerate}
	\item Vorbedingung: Benutzer mit Berechtigung Getränkeverwalter ist eingeloggt.
	\item Benutzer öffnet Menü für den Umsatzverlauf
	\item Benutzer wählt einen Trinker und einen Zeitraum auf, für den er den Umsatz ansehen möchte.
	\item Webinterface zeigt den ausgewählten Umsatz für den ausgewählten Zeitraum an.
\end{enumerate}

\end{enumerate}

\subsection{KANN-Anforderungen}

\begin{enumerate}[{USC}1]

\item \textbf{PayPal Anbindung}: Ich als Trinker möchte mein Guthaben per PayPal aufladen, weil ich nicht gern Bargeld mit mir Rumtrage und den Getränkeverwalter nicht nerven möchte.
\begin{enumerate}
	\item Vorbedingung: Benutzer mit Berechtigung Trinker ist eingeloggt.
	\item Benutzer öffnet Menü/Button um per PayPal zu bezahlen.
	\item Benutzer wählt Betrag, der aufgeladen werden soll und bezahlt diesen per PayPal
	\item Der aufgeladene Betrag wird dem Benutzer zugeschrieben und ist auf seinem Konto sichtbar.
\end{enumerate}

\end{enumerate}

\section{Optional: Non-Functional Requirements}
TODO: Löschen???

\section{Technologie-Stack}
\subsection{Datenhaltung}
Zur Datenhaltung soll die relationale Datenbank (TODO:PostrgresSQL?) genutzt werden.
\subsection{Backend}
Das Backend soll aus einer REST-API bestehen, welche die benötigten Daten im JSON-Format bereitstellt. Dafür soll Python mit dem Framework (TODO: FastAPI?) verwendet werden.
\subsection{Frontend}
Für die Darstellung der Daten und die Benutzerinteraktion im Browser soll das TODO: ReactJS? mit dem Framework TODO: Next.js? verwendet werden.
\subsection{Conatinerisierung}
TODO:
Um die Anwendung Cloud-Nativ zu gestalten, sollen Front- und Backend in einem Docker-Container laufen. Alle Docker-Container sollen mithilfe von Docker-Compose zu einer Anwendung zusammengefasst werden können.


%%% Previous TechReps
\nocite{ModA-TR-2023SS-WAE-TeamWeiss-Neunerln}
\nocite{ModA-TR-2023SS-BDCC-TeamRot-CompVisPipeline}
\nocite{ModA-TR-2023SS-BDCC-TeamBlau-NauticalNonsense}
\nocite{ModA-TR-2023SS-BCN-TeamGruen-TorpedoTactics}
\nocite{ModA-TR-2023SS-BCN-TeamCyan-Stockbird}
\nocite{ModA-TR-2023SS-BCN-TeamBlau-FancyChess}
\nocite{ModA-TR-2023WS-SWT-TeamRot-SGDb}
\nocite{ModA-TR-2023WS-SWT-TeamGruen-OPCUANetzwerk}
\nocite{ModA-TR-2022SS-WAE-TeamWeiss-WoIstMeinGeld}
\nocite{ModA-TR-2022SS-BDCC-TeamWeiss-TwitterDash}
\nocite{ModA-TR-2022SS-BDCC-TeamRot-Reddiment}
\nocite{ModA-TR-2022SS-BDCC-TeamGruen-ExplosionGuy}
\nocite{ModA-TR-2022SS-BDCC-TeamCyan-OTHWiki}
\nocite{ModA-TR-2022WS-SWT-TeamGruen-Graphvio}
\nocite{ModA-TR-2021SS-WAE-TeamWeiss-CovidDashboard}
\nocite{ModA-TR-2021SS-WAE-TeamRot-FireForceDefense}
\nocite{ModA-TR-2021SS-WAE-TeamGruen-MedPlanner}

% ======== References =========
\sloppy
\printbibliography[notcategory=selfref]

\end{document}
